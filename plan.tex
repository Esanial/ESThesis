\documentclass[a4paper,notitlepage]{article}
\usepackage[utf8]{inputenc}
\usepackage[francais]{babel}
\usepackage[]{geometry}

\title{Plan de thèse}
\author{Elsa Sanial}
\date{08/01/2018}


\usepackage{Sweave}
\begin{document}
\input{plan-concordance}
\maketitle


\textit{NB: ajouter des parties de discussion conclusive en fin de chaque partie
reformuler les titres
et conceptualiser les transitions entre chaque partie.} 

\part{Partie Introductive}
\vspace{1cm}
\textit{Justification du sujet et de l'approche transdisciplinaire nourrie par la Political ecology et les évolutions récentes des théories sur la relation population/environnement appliquées au contexte post-forestier du Sud Ivoirien}


\part{Changements environnementaux et effets des arbres associés sur le microclimat d'une cacaoyère}
\vspace{1cm}
\section{Chapitre 1 Etat de l'art: agroforesterie, changements environnementaux et résilience des cacaoyers}

\section{Chapitre 2 Méthode pour la mesure des évolutions climatiques et de l'effet d'arbres associés sur la résilience d'une cacaoyère en période de saison sèche}

\section{Chapitre 3 Résultats: Evolutions climatiques (1920-2000) et effets de la présence d'arbres associés sur le microclimat d'une cacaoyère}


\part{Déterminants environnementaux de la multifonctionnalité des agroforêts}
\vspace{1cm}
\section{Chapitre 1 Etat de l'art: les services fournis par l'agroforesterie cacao}

\section{Chapitre 2 Méthode pour la quantification des services potentiels et réalisés (biodiversité, carbone et valeurs d'usage) dans les systèmes agroforestiers cacao ivoiriens actuels}

\section{Chapitre 3 Résultats}

1. Service Biodiversité
2. Service Stockage de Carbone
3. Multifonctionnalité des arbres: valeurs d'usage
4. Evaluation du rôle des déterminants environnementaux dans le renforcement de ces services

\textit{Dans telle condition environnementale, ce que l'on peut espérer comme service a été démontré. Quels sont les facteurs qui expliquent que ce potentiel n'est pas réalisé? --> plutôt utiliser le terme de stratégie}

\part{Facteurs et obstacles à l'introduction d'arbres associés: stratégies des producteurs pour un cacao post-forestier}
\vspace{1cm}
\section{Chapitre 1 Etat de l'art: dynamiques agroforestières et cycles du cacao, la Côte d'Ivoire en marge des travaux}

\section{Chapitre 2 Méthode}

\section{Chapitre 3 Résultats: Approche qualitative à travers deux monographies}

\subsection{Blé: la reconfiguration d'un village autochtone par l'économie de plantation et la disparition corollaire de la forêt}

1. Evolution des usages du sol 1956-2017: une mosaïque d'usages agricoles après la spécialisation café/cacao

2. Caractéristiques économiques, modes d'allocations des terres, dépendance au cacao et usages des produits forestiers

3. Dynamiques agroforestières actuelles

4. Facteurs et obstacles, stratégies alternatives à la réintroduction d'arbres dans les champs

-------------------------------------------------------------------

\subsection{Kragui, quel avenir pour les villages de migrants issus des fronts pionniers des années 70 dans l'Ouest ivoirien?}

1. Evolution des usages du sol 1956-2017: de la forêt à la saturation foncière

2. Caractéristiques économiques, modes d'allocation des terres, dépendance au cacao et usages des produits forestiers

3. Dynamiques agroforestières

4. Facteurs et obstacles à la réintroduction d'arbres dans les champs

\section{Chapitre 4 Résultats: Approche quantitative: modélisations des facteurs et obstacles à l'adoption de l'agroforesterie cacao}


\part{Discussion conclusive}



\end{document}
